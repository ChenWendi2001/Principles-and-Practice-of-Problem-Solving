%!TEX program = pdflatex
\documentclass[a4paper]{article}
\title{\textbf{Homework 7}}
\author{Wendi Chen}
\date{}
\usepackage{listings}
\usepackage{graphicx} 
\usepackage{indentfirst} 
\usepackage{algorithm}
\usepackage{algorithmicx}
\usepackage{algpseudocode}
\usepackage{amsmath}
\newtheorem{proof}{Proof}[section]
\usepackage{algorithm,algpseudocode,float}
\usepackage{lipsum}

\makeatletter
\newenvironment{breakablealgorithm}
  {% \begin{breakablealgorithm}
   \begin{center}
     \refstepcounter{algorithm}% New algorithm
     \hrule height.8pt depth0pt \kern2pt% \@fs@pre for \@fs@ruled
     \renewcommand{\caption}[2][\relax]{% Make a new \caption
       {\raggedright\textbf{\ALG@name~\thealgorithm} ##2\par}%
       \ifx\relax##1\relax % #1 is \relax
         \addcontentsline{loa}{algorithm}{\protect\numberline{\thealgorithm}##2}%
       \else % #1 is not \relax
         \addcontentsline{loa}{algorithm}{\protect\numberline{\thealgorithm}##1}%
       \fi
       \kern2pt\hrule\kern2pt
     }
  }{% \end{breakablealgorithm}
     \kern2pt\hrule\relax% \@fs@post for \@fs@ruled
   \end{center}
  }
\makeatother
\begin{document}
\maketitle

\section{Bonus: Edit distance}
\subsection{Problem analysis}
As the book describes, the individual costs of
the copy and replace operations are less than the combined costs of the delete and
insert operations.
Similarily, we can also assert that the individual cost of copy is
less than the individual cost of replace, otherwise the copy operation
would not be used.\\

It's naturally to find that this is a problem that can be divided
into subproblems that overlap. So dynamic programming is a good way to solve it.
We assume $c[i,j](0 \leq i \leq m, 0 \leq j \leq n)$ represents the least operation cost of tansforming
$x[1..i]$ to $y[1..j]$. At the same time, we assume $cost[i] (1 \leq i \leq 6)$ represents
the cost of \textbf{Copy, Replace, Delete, Insert, Twiddle, Kill} respectively and
$op[i,j](0 \leq i \leq m, 0 \leq j \leq n)$ for recording the last operation when
tansforming $x[1..i]$ to $y[1..j]$.
\subsection{Pseudo-code}
\renewcommand{\algorithmicrequire}{\textbf{Input:}}  % Use Input in the format of Algorithm
\renewcommand{\algorithmicensure}{\textbf{Output:}} % Use Output in the format of Algorithm
\begin{breakablealgorithm}
    \caption{Pseudocode of the Dynamic Programming to Solve Edit Distance}
    \begin{algorithmic}[1]
        \Require
        $x, y$: two input strings
        \Ensure
        $c, op$: two 2-D arrays
        $num$: the number of characters in x that has been examined
        \State create $c[0..m, 0..n], op[0..m, 0..n]$
        \State create $current\_cost[0..6]$
        \State set $c[0, 0] = 0$
        \State set $op[0, 0] = 0$
        \State set $num = 0$
        \For{$i = 1$ to $m$}
        \State $c[i,0] = cost[3] + c[i-1,0]$
        \State $op[i,0] = 3$
        \EndFor
        \For{$i = 1$ to $n$}
        \State $c[0,n] = cost[4] + c[0,i-1]$
        \State $op[0,i] = 4$
        \EndFor
        \If{$n==0$ and $cost[6]<c[m,0]$}
        \State $c[m,0] = cost[6]$
        \State $op[m,0] = 6$
        \EndIf
        \For{$i = 1$ to $m$}
        \For{$j = 1$ to $n$}
        \For{$k = 1$ to $6$}
        \State$current\_cost[k] = \infty$
        \EndFor
        \If{$x[i]==y[j]$}
        \State $current\_cost[1] = cost[1] + c[i-1][j-1]$
        \Else
        \State $current\_cost[2] = cost[2] + c[i-1][j-1]$
        \EndIf
        \State$current\_cost[3] = cost[3] + c[i-1][j]$
        \State$current\_cost[4] = cost[4] + c[i][j-1]$
        \If{$2 \leq i$ and $2 \leq j$ and $x[i-1] == y[j]$ and $x[i] == y[j-1]$}
        \State $current\_cost[5] = cost[5] + c[i-2][j-2]$
        \EndIf
        \If{$i == m$ and $j == n$}
        \For{$k = 1$ to $m-1$}
        \If{$cost[6] + c[k,n] < current\_cost[6]$}
        \State $current\_cost[6] = cost[6] + c[k,n]$
        \State $num = k$
        \EndIf
        \EndFor
        \EndIf
        \State $c[i,j] = \infty$
        \For{$k = 1$ to $6$}
        \If{$current\_cost[k] < c[i,j]$}
        \State $c[i,j] = current\_cost[k]$
        \State $op[i,j] = k$
        \EndIf
        \EndFor
        \EndFor
        \EndFor
        \State\Return{$c, op, num$}

    \end{algorithmic}
\end{breakablealgorithm}

\begin{breakablealgorithm}
    \caption{Pseudocode of Printing an Optimal Operation Sequence}
    \begin{algorithmic}[2]
        \Require
        $op$: operation 2-D array 
        $num$: the number of characters in x that has been examined
        
        \Function{print\_seq}{op, num, i, j}
        \If{$op[i,j]==0$}
        \State\Return
        \ElsIf{$op[i,j]==1$}
        \State PRINT\_SEQ(op, num, i-1, j-1)
        \State \textbf{print} copy
        \ElsIf{$op[i,j]==2$}
        \State PRINT\_SEQ(op, num, i-1, j-1)
        \State \textbf{print} replace
        \ElsIf{$op[i,j]==3$}
        \State PRINT\_SEQ(op, num, i-1, j)
        \State \textbf{print} delete
        \ElsIf{$op[i,j]==4$}
        \State PRINT\_SEQ(op, num, i, j-1)
        \State \textbf{print} insert
        \ElsIf{$op[i,j]==5$}
        \State PRINT\_SEQ(op, num, i-2, j-2)
        \State \textbf{print} twiddle
        \ElsIf{$op[i,j]==6$}
        \State PRINT\_SEQ(op, num, num, j)
        \State \textbf{print} kill
        \EndIf
        \EndFunction

    \end{algorithmic}
\end{breakablealgorithm}

\subsection{Complexity}
From these two nested loops we can get the time 
complexity and the space complexity of the whole algorithm 
are both $O(mn)$.

\end{document}


